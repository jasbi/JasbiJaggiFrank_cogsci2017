% Template for Cogsci submission with R Markdown

% Stuff changed from original Markdown PLOS Template
\documentclass[10pt, letterpaper]{article}

\usepackage{cogsci}
\usepackage{pslatex}
\usepackage{float}
\usepackage{caption}

% amsmath package, useful for mathematical formulas
\usepackage{amsmath}

% amssymb package, useful for mathematical symbols
\usepackage{amssymb}

% hyperref package, useful for hyperlinks
\usepackage{hyperref}

% graphicx package, useful for including eps and pdf graphics
% include graphics with the command \includegraphics
\usepackage{graphicx}

% Sweave(-like)
\usepackage{fancyvrb}
\DefineVerbatimEnvironment{Sinput}{Verbatim}{fontshape=sl}
\DefineVerbatimEnvironment{Soutput}{Verbatim}{}
\DefineVerbatimEnvironment{Scode}{Verbatim}{fontshape=sl}
\newenvironment{Schunk}{}{}
\DefineVerbatimEnvironment{Code}{Verbatim}{}
\DefineVerbatimEnvironment{CodeInput}{Verbatim}{fontshape=sl}
\DefineVerbatimEnvironment{CodeOutput}{Verbatim}{}
\newenvironment{CodeChunk}{}{}

% cite package, to clean up citations in the main text. Do not remove.
\usepackage{cite}

\usepackage{color}

% Use doublespacing - comment out for single spacing
%\usepackage{setspace}
%\doublespacing


% % Text layout
% \topmargin 0.0cm
% \oddsidemargin 0.5cm
% \evensidemargin 0.5cm
% \textwidth 16cm
% \textheight 21cm

\title{Rapid Acquisition of Disjunction from Prosody and Consistency}


\author{{\large \bf Masoud Jasbi} \\ \texttt{masoudj@stanford.edu} \\ Department of Linguistics \\ Stanford University \And {\large \bf Akshay Jaggi} \\ \texttt{ajaggi@stanford.edu} \\ Department of Linguistics \\ Stanford University \And {\large \bf Michael C. Frank} \\ \texttt{mcfrank@stanford.edu} \\ Department of Psychology \\ Stanford University }

\begin{document}

\maketitle

\begin{abstract}
You need to mention the conflict between the usage-based accounts and
the nativist accounts. You need to say that the exclusivity of
disjunction can be learned from prosody and inconsistency of the
disjuncts themselves (e.g.~dirty or clean). Consistent with the nativist
account, we find that even though \emph{or} is rare in child-directed
speech compared to \emph{and}, it is learned relatively quickly by
children and produced at an adult rate by the age four. We also show
that consistent with usage-based accounts of learning, there is a
consistent pattern between usage and interpretation that children can
learn from.

\textbf{Keywords:}
language acquisition; word learning; logical words; and; or; nativism;
constructivism.
\end{abstract}

\section{Introduction}\label{introduction}

``James Bond producer says next 007 could be black \textbf{or} a
woman'', reported the social media and entertainment company LADbible in
a tweet. A twitter user named Robert responded sarcastically with: ``if
only women could be black!'' What in the producer's speech gave the
impression that the next 007 could not be both black \textbf{and} a
woman? The word \emph{or} is associated with two interpretations:
\textbf{inclusive}, and \textbf{exclusive}. A disjunction like ``A or
B'' is inclusive when it is interpreted as ``A or B or both''. This is
probably what LADbible meant when reporting the James Bond producer.
However, ``A or B'' can also be interpreted as exclusive: ``A or B but
\textbf{not both}''. Robert's response shows that he had an exclusive
interpration of \emph{or}. What factors determine the interpretation of
\emph{or} and how do children learn its meaning given this ambiguity?

A large body of research in linguistics and philosophy in the past 50
years has created a common consensus on the meaning of \emph{or} (see
Aloni (2016)). Data on the interpretation of disjunction across
different sentences, contexts, and even languages show that the core
meaning of disjunction words such as \emph{or} is \textbf{inclusive}.
This is similar to the definition of disjunction in formal logics. The
\textbf{exclusive} interpretation of \emph{or} is the result of
enhancing its inclusive semantics via other (extra-semantic) factors
such as intonation (Pruitt \& Roelofsen, 2013), inconsistency of the
options (Geurts, 2006), and pragmatic reasoning over the speaker's
choice of \emph{or} instead of \emph{and} (Grice, 1975). In other words,
interpreting a disjunciton is a complex process that needs to take into
account the meaning of \emph{or} as well as different structural and
contextual factors that accompany it. How do we, as children, learn such
a complex interpretive system?

There are two accounts of children's acquisition of disjunciton: a
constructivist account, and a nativist account. Under the constructivist
account, children learn the meaning of \emph{or} by paying attention to
how parents use it in different contexts. They form usage-rules and
expand their usage repertoire as they grow up. The prediction is that
children's production of \emph{or} in children mirrors what they hear
from parents. Morris (2008) found that \emph{or} is much less frequent
than \emph{and} in parents' speech to children. As predicted by the
constructivist account, children also learn to produce \emph{and}
quickly while for \emph{or} there is a gradual increase in production,
possibly reaching the adult level at age five or six. In children's
speech, \emph{and} appears quickly and around the age 2 or 3 while it
takes several years for \emph{or} to appear. The faster acquisition of
\emph{and} is consistent with the constructivist theory that emphasizes
frequency of usage in children's linguistic development. Morris also
showed that the majority of \emph{or} examples children hear are
exclusive. He showed that consistent with the constructivist account,
the majority of \emph{or}s children produce are also exclusive.

However, several comprehension studies of \emph{or} in different
linguistic contexts show that children between the ages of three and
five interpret \emph{or} as inclusive disjunction (Crain, 2012). This is
a surprising finding given the previous finding that the majority of
\emph{or} examples children hear are exclusive. Crain (2012) suggests
that children rely on innate knowledge that the the disjunciton operator
in their native language must have an inclusive meaning. More broadly,
logical nativism argues that children have an innate knowledge that the
connective words in their native langauge should refer to operators
similar to those in first order logic. Exclusive disjunction is not an
operator available

Here we provide alternative answers to two puzzles raised in previous
literature: 1. Production-Comprehension Discrepency: the production data
suggests that learning \emph{or} is a gradual and long process, possibly
reaching an adult-like level at the age five or six. However,
comprehension studies suggest that children have an adult-like
understanding of \emph{or} at the age of four or even three. 2.
Exclusivity puzzle: even though the majority of \emph{or} examples
children hear are exclusive, comprehension studies show that children
understand \emph{or} as inclusive disjunction. How do children learn
what they do not seem to hear?

The first study shows that despite (relatively) rare occurances of
\emph{or}, it is learned relatively quickly and between the ages of 2
and 4. The finding confirms the nativist

Implications for future work.

\section{Study 1: Corpus Study}\label{study-1-corpus-study}

First, we conducted an exploratory and large scale investigation of
\emph{and} and \emph{or} productions in parents and children. The goal
of the study was to find out when children start producing these words
and when they reach the adult rate of production. We conclude that
children start producing \emph{and} around 1.5 or 2 years of age, and
\emph{or} between the ages of 2 and 3. They reach the adult rate of
production for \emph{and} around 3 and for \emph{or} around 4 or
possibly earlier.

\subsection{Methods}\label{methods}

We accessed the Child Language Data Exchange System (CHILDES, MacWhinney
(2000)) via the online platform
\href{http://childes-db.stanford.edu/}{childes-db} and its associated R
package childesr (Sanchez et al., in prep). We extracted all instances
of \emph{and} and \emph{or} from the English corpora (ENG-NA and
ENG-UK). We limited our analysis to the data between one and six years
because there is scarce data outside this age range. We computated the
relative frequency of connective production by dividing the total number
of \emph{and}/\emph{or} in the speech of fathers, mothers, and children
at a particular age by the total number of words spoken at that age. We
present the relative frequency as parts per thousand.

\subsection{Results}\label{results}

In figure 1, we show the relative frequencies of \emph{and} and
\emph{or} in the speech of parents and children between one and six
years. It is important to note that the y-axes for \emph{and} vs.
\emph{or} show different ranges. This is due to the big difference in
the relative frequencies of \emph{and} and \emph{or}. In the speech of
parents, \emph{and} is produced around 20 times per thousand words while
\emph{or} is only produced around 2 times per thousand words. This
confirms previous findings that \emph{or} is much less frequent in child
directed speech than a similar funciton word such as \emph{and}.

\begin{CodeChunk}
\begin{figure}[H]
\includegraphics{figs/OverallConnectivePlots-1} \includegraphics{figs/OverallConnectivePlots-2} \caption[The relative frequency of AND (top) and OR (bottom) per thousand words in the speech of fathers, mothers, and children between the ages of 1 and 6]{The relative frequency of AND (top) and OR (bottom) per thousand words in the speech of fathers, mothers, and children between the ages of 1 and 6.}\label{fig:OverallConnectivePlots}
\end{figure}
\end{CodeChunk}

\emph{And} and \emph{or} seem to show different developmental
rajectories in the speech of children. For \emph{and}, there is a rapid
increase in its production between the ages of 1.5 and 3 before it
reaches the adult rate around the age 3 and stay at that level until the
age 6. For \emph{or}, on the other hand, we see a slow incrase from the
age 2 until the age 6 when it reaches the adult rate. This difference in
the development of \emph{and} \& \emph{or} production was attributed to
the frequency of these items in child-directed speech. Since \emph{and}
is much more frequent than \emph{or}, it is learned much faster than
\emph{or}. Morris (2008) argued that such patterns support the
item-based and usage-based acquisition of logical words.

However, the analysis above does not control for other factors that can
affect the production of words by children. An important factor to
control for is the development of speech acts. While content words such
as \emph{dog} or \emph{chair} may appear freely in different types of
speech acts, function words are highly constrained by the type of speech
acts they can appear in. For example, it is reasoable to assume that
question words such as \emph{how} and \emph{why} are much more likey to
occur in questions than statements (declaratives). If parent-child
interaction is such that parents ask more questions than children, it is
not surprising to find higher rates of \emph{how} and \emph{why}
production in parents than children. Therefore, it is important to
control for the speech act a function word appears in.

Figure 2 shows the relative frequencies of \emph{and} and \emph{or} in
questions vs.~declaratives, in the speech of parents and children
between one and six years. Here, the relative frequency is computed by
dividing the total number of \emph{and}/\emph{or} in a
question/declarative in the speech of fathers, mothers, and children at
a particular age, by the total number of words in a question/declarative
spoken at that age. As before, we present the relative frequency as
parts per thousand.

\begin{CodeChunk}
\begin{figure}[H]
\includegraphics{figs/byspeechActPlots-1} \includegraphics{figs/byspeechActPlots-2} \caption[The relative frequency of AND (top) and OR (bottom) per thousand words in delcaratives and questions in the speech of fathers, mothers, and children between the ages of 1 and 6,]{The relative frequency of AND (top) and OR (bottom) per thousand words in delcaratives and questions in the speech of fathers, mothers, and children between the ages of 1 and 6,}\label{fig:byspeechActPlots}
\end{figure}
\end{CodeChunk}

The results show similar developmental trajectories for the production
of \emph{and} and \emph{or} in children. For both words, there is a
relatively rapid incease in their frequency between the ages of 2 and 4
before reaching the parent rate at the age of 4 and staying at that rate
until the age of 6. This pattern of production is consistent with the
nativist observations that the acquisition of \emph{and} and \emph{or}
is rapid and that children have an adult-like comprehension of these two
connectives at the age 4.

\subsection{Summary}\label{summary}

And is a lot more frequent than or in child directed speech.

In the first six years, it appears that children reach the adult
production rate for and but not or.

This is at least partly because or is more frequent in questions and
children produce fewer questions than parents.

The developmental trajectory of connective production is best described
as a quick increase in production between the ages of 2 and 4 and
staying around the parents' rate between the ages 4 and 6. This is
compatible with the comprehension studies which suggest children
understand \emph{and} and \emph{or} by the age four.

\section{Study 2: Annotation Study}\label{study-2-annotation-study}

Use standard APA citation format. Citations within the text should
include the author's last name and year. If the authors' names are
included in the sentence, place only the year in parentheses, as in
({\textbf{???}}), but otherwise place the entire reference in
parentheses with the authors and year separated by a comma
({\textbf{???}}). List multiple references alphabetically and separate
them by semicolons ({\textbf{???}}, {\textbf{???}}). Use the et. al.
construction only after listing all the authors to a publication in an
earlier reference and for citations with four or more authors.

For more information on citations in R Markdown, see
\textbf{\href{http://rmarkdown.rstudio.com/authoring_bibliographies_and_citations.html\#citations}{here}.}

\subsection{Footnotes}\label{footnotes}

Indicate footnotes with a number\footnote{Sample of the first
footnote.} in the text. Place the footnotes in 9 point type at the
bottom of the page on which they appear. Precede the footnote with a
horizontal rule.\footnote{Sample of the second footnote.}

\subsection{Figures}\label{figures}

All artwork must be very dark for purposes of reproduction and should
not be hand drawn. Number figures sequentially, placing the figure
number and caption, in 10 point, after the figure with one line space
above the caption and one line space below it. If necessary, leave extra
white space at the bottom of the page to avoid splitting the figure and
figure caption. You may float figures to the top or bottom of a column,
or set wide figures across both columns.

\subsection{Two-column images}\label{two-column-images}

You can read local images using png package for example and plot it like
a regular plot using grid.raster from the grid package. With this method
you have full control of the size of your image. \textbf{Note: Image
must be in .png file format for the readPNG function to work.}

You might want to display a wide figure across both columns. To do this,
you change the \texttt{fig.env} chunk option to \texttt{figure*}. To
align the image in the center of the page, set \texttt{fig.align} option
to \texttt{center}. To format the width of your caption text, you set
the \texttt{num.cols.cap} option to \texttt{2}.

\begin{CodeChunk}
\begin{figure*}[h]

{\centering \includegraphics{figs/2-col-image-1} 

}

\caption[This image spans both columns]{This image spans both columns. And the caption text is limited to 0.8 of the width of the document.}\label{fig:2-col-image}
\end{figure*}
\end{CodeChunk}

\subsection{One-column images}\label{one-column-images}

Single column is the default option, but if you want set it explicitly,
set \texttt{fig.env} to \texttt{figure}. Notice that the
\texttt{num.cols} option for the caption width is set to \texttt{1}.

\begin{CodeChunk}
\begin{figure}[H]

{\centering \includegraphics{figs/image-1} 

}

\caption[One column image]{One column image.}\label{fig:image}
\end{figure}
\end{CodeChunk}

\subsection{R Plots}\label{r-plots}

You can use R chunks directly to plot graphs. And you can use latex
floats in the fig.pos chunk option to have more control over the
location of your plot on the page. For more information on latex
placement specifiers see
\textbf{\href{https://en.wikibooks.org/wiki/LaTeX/Floats,_Figures_and_Captions}{here}}

\begin{CodeChunk}
\begin{figure}[H]

{\centering \includegraphics{figs/plot-1} 

}

\caption[R plot]{R plot}\label{fig:plot}
\end{figure}
\end{CodeChunk}

\subsection{Tables}\label{tables}

Number tables consecutively; place the table number and title (in 10
point) above the table with one line space above the caption and one
line space below it, as in Table 1. You may float tables to the top or
bottom of a column, set wide tables across both columns.

You can use the xtable function in the xtable package.

\begin{table}[H]
\centering
\begin{tabular}{rrrrr}
  \hline
 & Estimate & Std. Error & t value & Pr($>$$|$t$|$) \\ 
  \hline
(Intercept) & 0.06 & 0.11 & 0.6 & 0.57 \\ 
  x & 1.95 & 0.11 & 18.5 & 0.00 \\ 
   \hline
\end{tabular}
\caption{This table prints across one column.} 
\end{table}

\section{Conclusion (Masoud)}\label{conclusion-masoud}

List what you showed and argued for.

Talk about future directions.

Possibly talk about the original tweet and how that falls still outside
what your algorithm learns. \# Acknowledgements

Place acknowledgments (including funding information) in a section at
the end of the paper.

\section{References}\label{references}

\setlength{\parindent}{-0.1in} \setlength{\leftskip}{0.125in} \noindent

\hypertarget{refs}{}
\hypertarget{ref-Aloni2016}{}
Aloni, M. (2016). Disjunction. In E. N. Zalta (Ed.), \emph{The stanford
encyclopedia of philosophy}. Stanford University. Retrieved from
\url{https://plato.stanford.edu/archives/win2016/entries/disjunction/}

\hypertarget{ref-crain2012emergence}{}
Crain, S. (2012). \emph{The emergence of meaning}. Cambridge University
Press.

\hypertarget{ref-geurts2006exclusive}{}
Geurts, B. (2006). Exclusive disjunction without implicatures.
\emph{Ms., University of Nijmegen}.

\hypertarget{ref-grice1975logicconvo}{}
Grice, H. P. (1975). Logic and conversation. In P. Cole \& J. Morgan
(Eds.), \emph{Syntax and semantics} (Vol. 3: Speech Acts, pp. 43--58).
Academic Press.

\hypertarget{ref-macwhinney2000childes}{}
MacWhinney, B. (2000). \emph{The childes project: The database} (Vol.
2). Psychology Press.

\hypertarget{ref-morris2008logically}{}
Morris, B. J. (2008). Logically speaking: Evidence for item-based
acquisition of the connectives and \&amp; or. \emph{Journal of Cognition
and Development}, \emph{9}(1), 67--88.

\hypertarget{ref-pruitt2013interpretation}{}
Pruitt, K., \& Roelofsen, F. (2013). The interpretation of prosody in
disjunctive questions. \emph{Linguistic Inquiry}, \emph{44}(4),
632--650.

\hypertarget{ref-childesdb}{}
Sanchez, A., Meylan, S., Braginsky, M., MacDonald, K., Yurovsky, D., \&
Frank, M. C. (in prep). Childes-db: A flexible and reproducible
interface to the child language data exchange system (childes).

\end{document}
